\documentclass[titlepage]{article}
\usepackage{outline}
\usepackage{pmgraph}
\usepackage[normalem]{ulem}
\usepackage{comment} % enables the use of multi-line comments (\ifx \fi)
\usepackage{lipsum} %This package just generates Lorem Ipsum filler text.
\usepackage{fullpage} % changes the margin
\title{\textbf{PreLab 07: Introduction to Behavioral Verilog and Logic
Synthesis}}
\author{Joseph Martinsen \\ ECEN 248-510 \\ TA: Michael Bass}
\date{\today}
%--------------------Make usable space all of page
\setlength{\oddsidemargin}{0in}
\setlength{\evensidemargin}{0in}
\setlength{\topmargin}{0in}
\setlength{\headsep}{-.25in}
\setlength{\textwidth}{6.5in}
\setlength{\textheight}{8.5in}
%--------------------Indention
\setlength{\parindent}{1cm}

\begin{document}
%--------------------Title Page
\maketitle

\ifx

\section*{Objectives}
\textit{Provide three or four major points you feel the lab is attempting to
teach you. Do NOT simply repeat the wording in the lab manual. Remember, you
should write about what you will learn, not what you will do. Keep in mind
these are not necessarily the same.}

\section*{Design}

\textit{This section contains the steps required to successfully complete the
lab along with all necessary diagrams, schematics, tables, equations, K-maps,
source code (for later labs) etc. Each of the aforementioned items in the design
section should include a written description contained within the body of the
text and should be labeled properly. Simply including a circuit schematic is
not sufficient! Note: As part of the design process for the first four labs,
you will create gate level schematics for each design. The schematics must be
complete such that the design can be correctly implemented by referencing only
the schematic. For the pre-lab, schematics may be hand drawn. However, for the
post-lab, schematics must be drawn on the computer using your choice of
drawing program. Your TA can recommend freely available software packages if
 you do not have access to a drawing program}

\section*{Results}

\textit{Use this section to discuss the observations you made during the lab.
 Include a comparison of what you expected to what you actually observed and
 provide an in-depth discussion of why you feel the circuit you built behaved
 the way it did. Like the design section, include diagrams and tables where
 appropriate. Be sure to label them properly and provide an adequate description
  in the body of the text.}

\section*{Conclutions}

\textit{This section should briefly summarize what it was that you did in lab
and provide some insight into what you learned. Additionally, you should mention
 skills you acquired during the execution of the lab assignment and discuss how
 you might use these skills in future labs. This portion of lab should tie into
 the objectives you talked about at the beginning of lab.}

\section*{Questions}

\textit{Finally, you must thoroughly answer the questions provided in each lab.
These questions are presented to you in order to test your understanding of the
topics in the lab. Brief, one-line answers are NOT acceptable!}

\begin{enumerate}
  \item \textbf{Question 1} \\
    Answer 1
  \item \textbf{Question 2} \\
    Answer 2
\end{enumerate}

\section*{Attachments}
%Make sure to change these
Lab Notes, HelloWorld.ic, FooBar.ic
%\fi %comment me out

\begin{thebibliography}{9}
\bibitem{Robotics} Fred G. Martin \emph{Robotics Explorations: A Hands-On Introduction to Engineering}. New Jersey: Prentice Hall.
\bibitem{Flueck}  Flueck, Alexander J. 2005. \emph{ECE 100}[online]. Chicago: Illinois Institute of Technology, Electrical and Computer Engineering Department, 2005 [cited 30
August 2005]. Available from World Wide Web: (http://www.ece.iit.edu/~flueck/ece100).
\end{thebibliography}

%How to cite
Put your Problem statement here! Example of a Citation\cite[p.219]{Robotics}. Here's Another Citation\cite{Flueck}
\fi
\end{document}
