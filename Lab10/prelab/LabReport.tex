\documentclass[a4paper,12pt]{article}
\usepackage{outline}
\usepackage{pmgraph}
\usepackage[normalem]{ulem}
\usepackage{comment} % enables the use of multi-line comments (\ifx \fi)
\usepackage{lipsum} %This package just generates Lorem Ipsum filler text.
\usepackage{fullpage} % changes the margin
\usepackage{listings}
\usepackage{color}
\usepackage{mdframed}
\usepackage{listings}
\usepackage{amssymb}
\usepackage{amsmath}
\usepackage{graphicx}
\graphicspath{ {../} }
\renewcommand{\lstlistingname}{Code Block}% Listing -> Algorithm
\renewcommand{\lstlistlistingname}{List of \lstlistingname s}% List of Listings -> List of Algorithms

\linespread{1.5}
%--------------------Indention
\setlength{\parindent}{15pt}
\lstset{frame=shadowbox, rulesepcolor=\color{white}}
\mdfsetup{frametitlealignment=\center}
\lstset{
  numbers=left,
  stepnumber=1,
  firstnumber=1,
  numberfirstline=true
}

\begin{document}
\section*{Prelab Questions}

% Verilog code template
% \lstinputlisting[language=Verilog,,caption=4-Bit ALU ]{../code name.v}

\begin{enumerate}
  \item \textbf{With \underline{dataflow} Verilog, describe the Generate/Propagate Unit, the Carry-Lookahead Unit, and the Summation Unit in Figure 1 as separate modules. 
  Do \underline{not} include delays in your modules. We will add them later in the lab experiments.}
  
    \lstinputlisting[language=Verilog,,caption=Generate Propagate Unit]{generate_propagate_unit.v}
    
    \lstinputlisting[language=Verilog,,caption=Carry Lookahead Unit]{carry_lookahead_unit.v}
    
    \lstinputlisting[language=Verilog,,caption=Summation Unit]{summation_unit.v}

  \item \textbf{Now, use structural Verilog along with the modules you have just created to wire up a 4-bit Carry-Lookahead adder.}
  
  \lstinputlisting[language=Verilog,,caption=Carry Lookahead 4-Bit]{carry_lookahead_4bit.v}


  
  \item \textbf{What is the gate count of your 4-bit \textit{carry-lookahead} adder?}
  
  \begin{table}[h]
\centering
% \caption{Gate Count}
  \begin{tabular}{l l}
  Module & \# of Gates \\ \hline \hline
  gpu0   & 8           \\ 
  gpu01  & 8           \\ 
  gpu2   & 8           \\ 
  gpu3   & 8           \\ 
  clu0   & 8           \\ 
  clu1   & 8           \\ 
  clu2   & 8           \\ 
  clu3   & 8           \\ 
  su0    & 4           \\ 
  su1    & 4           \\ 
  su2    & 4           \\ 
  su3    & 4           \\ \hline
  \textbf{Total}  & \textbf{80}        
  \end{tabular}
\end{table}
  
  \item \textbf{The previous problems were concerned with a single-level 4-bit \textit{carry-lookahead} adder. In one of the lab experiments, we will construct a 16-bit, 2-level \textit{carry-lookahead} adder. The following questions will prepare you for this exercise. What is the propagation delay of the 16-bit, 2-\textit{level} carry-lookahead adder in Figure 2? Likewise, what is the gate-count?}

\end{enumerate}

\ifx
\begin{thebibliography}{1}
\bibitem{Verilog} Charles Kime \& Thomas Kaminski  \emph{Logic and Computer Design Fundamentals} \\ \hspace{15pt}\textit{http://www.cs.bilkent.edu.tr/~will/courses/CS223/Verilog/LCDF3_Verilog_Ch_4.pdf}
\end{thebibliography}

\section*{Attachments}
%Make sure to change these
Lab Notes, HelloWorld.ic, FooBar.ic
%\fi %comment me out

\begin{thebibliography}{9}
\bibitem{Verilog} Charles Kime & Thomas Kaminski  \emph{Logic and Computer Design Fundamentals} \textit{http://www.cs.bilkent.edu.tr/~will/courses/CS223/Verilog/LCDF3_Verilog_Ch_4.pdf}
\end{thebibliography}

%How to cite
Put your Problem statement here! Example of a Citation\cite[p.219]{Robotics}. Here's Another Citation\cite{Flueck}
\fi
\end{document}
