\documentclass[a4paper,12pt]{article}
\usepackage{outline}
\usepackage{pmgraph}
\usepackage[normalem]{ulem}
\usepackage{comment} % enables the use of multi-line comments (\ifx \fi)
\usepackage{lipsum} %This package just generates Lorem Ipsum filler text.
\usepackage{fullpage} % changes the margin
\usepackage{listings}
\usepackage{color}
\usepackage{mdframed}
\usepackage{amssymb}
\usepackage{listings}
\usepackage{amsmath}
\usepackage{graphicx}
\graphicspath{ {../} }
\renewcommand{\lstlistingname}{Code Block}% Listing -> Algorithm
\renewcommand{\lstlistlistingname}{List of \lstlistingname s}% List of Listings -> List of Algorithms

\linespread{1.5}
%--------------------Indention
\setlength{\parindent}{15pt}
\lstset{frame=shadowbox, rulesepcolor=\color{white}}
\mdfsetup{frametitlealignment=\center}
\lstset{
  numbers=left,
  stepnumber=1,
  firstnumber=1,
  numberfirstline=true
}

\begin{document}

\section*{PreLab Questions}

\begin{enumerate}
  \item \textbf{Supposed we want to build a 32-bit counter using the circuit in Figure 1. If the gates used to construct
  the half-adders in the circuit are assumed to have a 2ns delay each and the flip-flop overhead is
  assumed to be negligible, what is the maximum clock frequency we could use to drive our counter?
  Given the clock frequency you just calculated, how long would it take this counter to roll over (i.e.
  return to 0). Please show your calculations.}
  \begin{align*}
    \text{Since each gate has propagation}&\text{ delay of } 2ns \\
    \therefore \text{Total delay } (T) &= (2\cdot32)ns =64 ns \\
    \text{Maximum clock frequency } (f) &=\frac{1}{T}=\frac{1}{64}GHz=15MHz \\
    \text{Time taken for roll over } &=2^{32}=4.29s
  \end{align*}

  \item \textbf{Consider the use of a counter to divide an incoming clock. If our incoming clock signal is 32.768 kHz
  and we need a 64 Hz signal, how many bits would our counter need to have (i.e. what is n) to divide
  the incoming clock correctly.}
\begin{align*}
  \frac{32768}{64}&=512 \\
  \therefore \text{No. of bits } &=\log_{2}512=9
\end{align*}

  \item \textbf{If the Seconds per Division setting for Figure 3 was set to 1 ms/div, what do you think would be a
  good bit width for the counter in Figure 4, assuming a 50 MHz clock signal was driving the counter?
  Explain your answer.}
  \vspace{10pt}

\end{enumerate}

\ifx
\begin{thebibliography}{1}
\bibitem{Verilog} Charles Kime \& Thomas Kaminski  \emph{Logic and Computer Design Fundamentals} \\ \hspace{15pt}\textit{http://www.cs.bilkent.edu.tr/~will/courses/CS223/Verilog/LCDF3_Verilog_Ch_4.pdf}
\end{thebibliography}

\section*{Attachments}
%Make sure to change these
Lab Notes, HelloWorld.ic, FooBar.ic
%\fi %comment me out

\begin{thebibliography}{9}
\bibitem{Verilog} Charles Kime & Thomas Kaminski  \emph{Logic and Computer Design Fundamentals} \textit{http://www.cs.bilkent.edu.tr/~will/courses/CS223/Verilog/LCDF3_Verilog_Ch_4.pdf}
\end{thebibliography}

%How to cite
Put your Problem statement here! Example of a Citation\cite[p.219]{Robotics}. Here's Another Citation\cite{Flueck}
\fi
\end{document}
