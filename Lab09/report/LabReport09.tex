\documentclass[a4paper,12pt]{article}
\usepackage{outline}
\usepackage{pmgraph}
\usepackage[normalem]{ulem}
\usepackage{comment} % enables the use of multi-line comments (\ifx \fi)
\usepackage{lipsum} %This package just generates Lorem Ipsum filler text.
\usepackage{fullpage} % changes the margin
\usepackage{listings}
\usepackage{color}
\usepackage{mdframed}
\usepackage{listings}
\usepackage{amssymb}
\usepackage{amsmath}
\usepackage{graphicx}
\graphicspath{ {../screenshots/} }
\renewcommand{\lstlistingname}{Code Block}% Listing -> Algorithm
\renewcommand{\lstlistlistingname}{List of \lstlistingname s}% List of Listings -> List of Algorithms

\linespread{1.5}
%--------------------Indention
\setlength{\parindent}{15pt}
\lstset{frame=shadowbox, rulesepcolor=\color{white}}
\mdfsetup{frametitlealignment=\center}
\lstset{
  numbers=left,
  stepnumber=1,
  firstnumber=1,
  numberfirstline=true
}

\begin{document}
\section*{Objective}

  \hspace{15pt}The purpose of this lab was to expand on the students familiarity of sequential circuits
  by exposing them to the inner workings of a binary counter. The lab manual guided the students
  through desgining a binary up-counter using Verilog. Once designing was complete, the design
  will be synthesized onto the FPGA board using a student edited UCF file. Finally, the lab
  will conclude with two important use cases for binary counters, namely clock frequency division
  and I/O debouncing.

\section*{Design}
% The order is
% clock_divider.v
% 4 seperate clock signals image
% half_adder.v
% up_counter.v
% up countere image
% clock divider was edited so Count was 26 bits wide
% top_level.v
% top_level.ucf
% swich_bounce.v
% noDebounce.v
% withDebounce.v

  \lstinputlisting[language=Verilog,,caption=Clock Divider ]{../code/clock_divider.v}

  \lstinputlisting[language=Verilog,,caption=Half Adder ]{../code/half_adder.v}

  \lstinputlisting[language=Verilog,,caption=Up Counter ]{../code/up_counter.v}

  \lstinputlisting[language=Verilog,,caption=Top Level (Verilog) ]{../code/top_level.v}

  \lstinputlisting[language=Verilog,,caption=Top Level UCF ]{../code/top_level.ucf}

  \lstinputlisting[language=Verilog,,caption=Switch Bounce ]{../code/switch_bounce.v}

  \lstinputlisting[language=Verilog,,caption=No Debounce ]{../code/noDebounce.v}

  \lstinputlisting[language=Verilog,,caption=With Debounce ]{../code/withDebounce.v}

\section*{Results}

Measure and record the period of each clock signal using the green and yellow markers. Based
on your measurements, what frequency do you think the input (Experiment 1.4.b)
2. Open up the test bench file and try to understand what is going on. You should see that the test
bench produces a Clk signal. What is the frequency of that signal? (Exp 2.2.b)
3. You should also see that the test bench holds the counter in reset for a specific interval of time.
How long is that interval? (Exp 2.2.c)
4. After reset is de-asserted, the test bench holds the enable LOW for some amount of time before
allowing the counter to run. How long is this time period? (Exp 2.2.d)
5. What is this maximum count value and what signal in the waveform could we use to know
exactly when the counter is going to roll over? (Exp 2.2.f)
6. If we use a 50MHz clock to drive our frequency divider, what rate will the most significant bit
of the divider oscillate at? (Exp 2.3.a)
7. Copy the waveform on the scope into your lab write-up. (Exp 3.1.j)
8. Does the design work as intended? Why or why not? (Exp 3.2.f)

% Figure with caption
\newpage
\begin{figure}[h]
  \begin{center}
    \includegraphics[scale=.1]{2_2_e.png}
    \caption{\textit{2-Bit 2:1 MUX Plots}}
  \end{center}
\end{figure}
\newpage
\begin{figure}[h]
  \begin{center}
    \includegraphics[scale=.1]{IMG_8620.JPG}
    \caption{\textit{2-Bit 2:1 MUX Plots}}
  \end{center}
\end{figure}
\newpage
\begin{figure}[h]
  \begin{center}
    \includegraphics[scale=.1]{IMG_8621.JPG}
    \caption{\textit{2-Bit 2:1 MUX Plots}}
  \end{center}
\end{figure}
\newpage
\begin{figure}[h]
  \begin{center}
    \includegraphics[scale=.1]{IMG_8622.JPG}
    \caption{\textit{2-Bit 2:1 MUX Plots}}
  \end{center}
\end{figure}
\newpage
\begin{figure}[h]
  \begin{center}
    \includegraphics[scale=.1]{IMG_8624.JPG}
    \caption{\textit{2-Bit 2:1 MUX Plots}}
  \end{center}
\end{figure}
\newpage
\begin{figure}[h]
  \begin{center}
    \includegraphics[scale=.1]{RollOver.png}
    \caption{\textit{2-Bit 2:1 MUX Plots}}
  \end{center}
\end{figure}
\newpage
\begin{figure}[h]
  \begin{center}
    \includegraphics[scale=.1]{up_Count_waveform.png}
    \caption{\textit{2-Bit 2:1 MUX Plots}}
  \end{center}
\end{figure}

\section*{Conclusions}


\section*{Questions}

\begin{enumerate}
  \item \textbf{Include the source code with comments for all modules in lab. You do not have to include test bench
  code. Code without comments will not be accepted!}

  \textit{In the report.}
  
  \item \textbf{Include any UCFs that you wrote or modified.}

  \textit{In the report.}
  
  \item \textbf{Include screenshots of all waveforms captured during simulation in addition to the test bench console
  output for each test bench simulation.}

  \textit{In the report.}

  \item \textbf{Answer all questions throughout the lab manual.}

  \textit{In the report.}
  
\end{enumerate}

\section*{Student Feedback}

\begin{enumerate}
  \item \textbf{What did you like most about the lab assignment and why? What did you like least about it and why?}
  \vspace{10pt}

  \item \textbf{Were there any section of the lab manual that were unclear? If so, what was unclear? Do you have any suggestions for improving the clarity?}
  \vspace{10pt}

  \item \textbf{What suggestions do you have to improve the overall lab assignment?}
  \vspace{10pt}

\end{enumerate}

\ifx
\begin{thebibliography}{1}
\bibitem{Verilog} Charles Kime \& Thomas Kaminski  \emph{Logic and Computer Design Fundamentals} \\ \hspace{15pt}\textit{http://www.cs.bilkent.edu.tr/~will/courses/CS223/Verilog/LCDF3_Verilog_Ch_4.pdf}
\end{thebibliography}

\section*{Attachments}
%Make sure to change these
Lab Notes, HelloWorld.ic, FooBar.ic
%\fi %comment me out

\begin{thebibliography}{9}
\bibitem{Verilog} Charles Kime & Thomas Kaminski  \emph{Logic and Computer Design Fundamentals} \textit{http://www.cs.bilkent.edu.tr/~will/courses/CS223/Verilog/LCDF3_Verilog_Ch_4.pdf}
\end{thebibliography}

%How to cite
Put your Problem statement here! Example of a Citation\cite[p.219]{Robotics}. Here's Another Citation\cite{Flueck}
\fi
\end{document}
